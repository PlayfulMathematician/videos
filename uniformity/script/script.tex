\documentclass{article}
\pagestyle{empty}
\usepackage{graphicx} 
\usepackage{blindtext}
\usepackage{endnotes}
\usepackage{url}
\let\footnote=\endnote 
\usepackage[backend=biber, style=verbose-trad1, autocite=footnote]{biblatex}\addbibresource{references.bib}
\title{Uniform Polyhedra}
\author{Playful Mathematician}
\date{July 2025}
\begin{document}
\maketitle
\section*{Chapter -1: Introduction}
Geometry is all about symmetry and generally the shapes that are studied the most have the most symmetry. One of these families of shapes is the uniform polyhedra. This is the story of one of the most important family of polyhedra. \footnote{I will exclude skewniforms and infinite tilings and honeycombs from this discussion. They are considerably outside the scope of this video. Also I will only include shapes that can be represented in 3d euclidean geometry. Additionally, compounds and non dyadic polyhedroids will be excluded. Double covers don't really count.} 
\section*{Chapter 0: What really IS a uniform polyhedron}
There exists a very special operation called an isometry. An isometry is an operation that preserves the distance between all pairs of points. Let's look in 2 dimensions. \autocite{wolframIsometry}\footnote{There are 5 times of symmetries in 2 dimensions. Rotations, Reflections, Translations, Identity and Glide Reflection. Reflections and Rotations are the only symmetries we will study.} Point A is two units away from Point B in this space. Now if we apply a 90 degree rotation around say this point here then Point A will move to Point A' and Point B will move to Point B'. Point A' is two units away from Point B'. Given how restrictive law, not any old transformation would work in this context. For example dilating twice the size about this point, won't work as all the distance is doubled. An extremely important thing to understand about isometriess is that the only thing that matters in distinguishing isometries is the result point. Consider an isometry that rotates 90 degrees counter clockwise around point C. Now consider an isometry that rotates 270 degrees clockwise around point C. See under the first isometry point A leads to the same the same point that the second isometry leads to. There exists one very interesting isometry. The identity isometry. Under this isometry, Point A goes to Point A.  Now consider this square, if you apply a 90 degree rotation around the center of this square. The square looks the exact same. Therefore this 90 degree rotation is known as a symmetry of the square \autocite{wolframSymmetry}. Every single shape you can think of has at least one symmetry, because the identity isometry is a symmetry of every single shape. A square is what is known as a polygon, as it is made of lines that all connect. The points that connect the lines are called vertices. The lines themselves are called edges. If there exists a symmetry for every pair of edges that when applied would lead one edge to another, the polygon is known as edge transitive. Similarly if there exists a symmetry for every pair of vertices that when applied would lead one vertex to another, the polygon is vertex transitive. A polygon that is vertex transitive and edge transitive is regular. Squares are regular. But so is this triangle. And this pentagon. There seems to be one regular polygon for every side count. But how does one generate a regular polygon. Here let's try with an example, suppose you wished to construct a regular 37-gon. First we must choose a center, let's pick this point. Call it Point A. Now we must pick a second Point, this one, let's call it Point B. Then we must rotate Point B, one thirty seventh of a revolution around A. Now connect B with the new point. If you keep doing this you eventually link back to point B. Congrats, you have now generated a 37-gon. But wait, what is a uniform polyhedron. A polyhedron is a set of polygons that are connected and closed. The polygons of the polyhedron are called faces. A uniform polyhedron is a vertex transitive with regular faces. Now that we have got all that out of the way. We must begin.
\section*{Chapter 1: The Platonic Solids}
Let's limit our search to convex polyhedra where each face has the same number of sides. If we do this the only thing that matters is the number of sides in each faces and the number of faces around each vertex. Lets start with triangles. Placing three regular triangles around a vertex form a polyhedra called a tetrahedron. Having four regular triangles around a vertex can form an shape with 8 faces called an octahedron. Placing 5 regular triangles around a vertex can form a polyhedra with 20 faces known as an icosahedron. Placing 6 triangles around a vertex on the other hand cannot form a polyhedra as the triangles cannot fold inward. Let's try with squares next, folding in 3 squares around a vertex forms a shape called a cube which has 6 faces. But 4 squares around a vertex isn't able to fold inward. With pentagons, three pentagons around a vertex form a polyhedra known as the dodecahedron which as 12 faces. Four pentagons cannot fold inward. Three hexagons around a vertex also cannot fold inward and that means we have exhausted our search. This collection of polyhedra is known as the platonic solids and our goal from here on out is to try and modify these solids while still preserving vertex transitivity.
\section*{Chapter 2: The Archimedean Solids}
An interesting operation one can apply to polyhedra is that of truncation, essentially truncation is the process of cutting of the corners of a polyhedra but in a specific way to preserve uniformity. Truncating a cube results in a shape formed from regular octagons and triangles known as a truncated cube. There are 6 octagonal faces and 8 triangular faces. Interestingly an ordinary cube has 6 faces and 8 vertices. This isn't a coincidence, by cutting off the vertices of a cube, the result is that each vertex has an associated triangle. The octagonal faces are constructed via the truncation of each of the 6 square faces. The interesting thing about the triangular faces as they are helpful in describing the cube. A cube isn't just a shape with square faces it is also a shape where if you cut off all the corners the corners become triangular. Given this relationship, the triangle is often called the vertex figure of the cube. As interesting as the vertex figure is there is also another idea called the vertex type. A cube has vertex type 4.4.4. Because there are three squares around each vertex. Whilst the truncated cube has vertex type 8.8.3 because each vertex has 2 octagons followed by a triangle around it. Lets go through the some vertex figures. A tetrahedron has a triangular vertex figure with triangular faces. An octahedron has a square vertex figure with triangular faces. A dodecahedron has a triangular vertex figure with pentagonal faces. An icosahedron has a pentagonal vertex figure with triangular faces. Truncating a tetrahedron has vertex type 6.6.3. Truncating a octahedron has vertex type 6.6.4. Truncating a icosahedron has vertex type 6.6.5.Truncating a dodecahedron has vertex type 10.10.3. There is also another interesting operation besides truncation. Instead of cutting off each vertex leading to equal length, cut off the vertex at the exact midpoint of the edge. Rectifying square faces results in a square faces. The rectification has a vertex type of 4.3.4.3. An interesting thing about rectification is that rectifying a polyhedra will result in a vertex type of (face).(verf).(face).(verf). Rectifying an octahedron will result in 3.4.3.4. This is the same as the rectification of the cube. To understand why, we need understand the dual operation. You may have noticed that an octahedron has triangular faces and a square vertex figure. A cube has square faces and a triangular vertex figure. You may notice that vertex figures and faces are swapped. The operation that swaps the vertex figure and the face is known as duality. The cube is the dual of the octahedron. The dual of an icosahedron is a dodecahedron. Additionally the tetrahedron its own dual. This brings us back to rectification. The rectification of a cube and octahedron are the same, so the rectification of the cube is called a cuboctahedron. The rectification of an icosahedron and dodecahedron is known as the icosidodecahedron. But what about the rectification of a tetrahedron. A tetrahedron has a triangular vertex figure and face which means that the rectification of a tetrahedron has vertex type of 3.3.3.3. The same as the octahedron. The rectification of a tetrahedron is an octahedron. Just to review we have discussed 7 solids, the 5 truncations of the 5 platonic solids as well as the cuboctahedron and icosidodecahedron. Another operation is known as cantellation or expansion. Essentially, we move the faces of a polyhedra outward like this until those rectangular faces become square. This leads to each of the faces of the polyhedra still being faces while, the vertices become vertex figures and the edges become squares. Naturally, this leads to a vertex arrangement of face.square.verf.square. The interesting thing about this is that is operation is that the cantellation of a polyhedra is the same as the cantellation of its dual. The cantellation of a cube as well as the octahedron is known as the small rhombicuboctahedron. Additionally the icosahedron and its dual the dodecahedron can be cantellated to form the small rhombicosidodecahedron. Now you might wonder what the cantellation of the tetrahedron is. It should be 4.3.4.3 but that is the same as the cuboctahedron. The tetrahedron's cantellation is the cuboctahedron. Bevelling is another operation. Essentially it involves truncating a shape, and moving the faces away from the origin. This leads to a vertex type of truncated verf.truncated face. 4. Interestingly, the bevel of the dual of a shape is it's bevel. The bevel is a cube and its dual the octahedron is called a great rhombicuboctahedron which has square, hexagonal and octagonal faces on it. The bevel of an icosahedron or its dual the dodecahedron is a great rhombicosidodecahedron. It has square, hexagonal and decagonal faces. The tetrahedron has triangular faces and triangular vertex figures. These faces and vertex figures truncated are hexagons. So the vertex type of a bevelled tetrahedron is 4.6.6. This is the same vertex type of a truncated octahedron. Again the tetrahedron happens to be useless. We now need to talk about snubification. Essentially snubification is like cantellation but the the faces have a slight tilt leading to the squares generated by the edges to become two triangles. This leads to a vertex arrangement of 3.3.verf.3.face. Again the dual of the snub is the snub. The snub of the dodecahedron, and its dual the icosahedron is known as the snub dodecahedron. The snub of the cube and its dual the octahedron is known as the snub cube. Interestingly the snub of tetrahedron is an icosahedron. This shows how utterly useless the tetrahedron is. An interesting property of the snub polyhedra is chirality. See the snub polyhedra each have a different mirror image. These mirror images are not the same as the original polyhedra in question. So it feels like there are actually 4 snub polyhedra based off of handedness. Although, the interesting thing about reflection is that doing a 3d reflection around a plane is the exact same thing as doing a 4d rotation.  So the enatiomorphs of the snub cube are really just one 4d rotation away and shouldn't be counted as distinct. Now let us talk about the elongated square gyrobicupola. So those are all 18 uniform polyhedra. 5 platonic solids, 5 truncations, 2 rectifications, 2 cantellations, 2 bevels, 2 snub polyhedra. One could count an additionally 2 polyhedra by including the enatiomorphs but those aren't counted as distinct. Additionally one could count the elongated square gyrobicupola, but that is not uniform. So that is it. Right? Right??
\section*{Chapter 3: Prisms and Antiprisms}

\section*{Chapter 4: Kepler-Poinsot Polyhedra}
1. Explain pentagons
2. Explain Small stellated dodecahedron
3. Explain dual
4. Explain gissid
5. Explain dual
 \section*{Chapter 5: More Prisms and Antiprisms}
1. Prisms
2. Antiprisms
3. Retroprisms


\section*{Chapter 6: Quasiregular Polyhedra}
A quasi regular polyhedron is a polyhedron whose vertex type alternates. So for example the cuboctahedron has vertex type 4.3.4.3. It alternates between squares and triangles. The icosidodecahedorn also satisfies this property having vertex type 5.3.5.3. The octahedron (albeit we are stretching the definition here) technically satisfies the property with vertex figure 3.3.3.3. All of these shapes are rectifications of other shapes, the octahedron the rectification of the tetrahedron, the icosidodecahedron the rectification of the icosahedron and dodecahedron and the cuboctahedron is a rectification of the cube and octahedron. So what if we rectify the Kepler-Poinsot Polyhedra. The rectification of the great dodecahedron and its dual the small stellated dodecahedron is a pretty cool shape called the dodecadodecahedron. Additionally the rectification of the great icosahedron and the great stellated dodecahedron is known as the Great Icosidodecahedron. Now an important property that every polyhedra must satisfy is that of dyadicity. Basically every edge must have exactly two faces on it. Another important property is that of not being a compound, a compound is when you take a collection of polyhedra and just shove them together. So this collection of 5 cubes arranged symettrically is a compound is thus must be excluded from the set of uniform polyhedra. If you replace the Dodecahedron's faces with pentagrams and then place triangles you construct a shape called a Small ditrigonal icosidodecahedron. Now instead of triangles you can use pentagons which results in the ditrigonary dodecadodecahedron. If you remove the pentagrams and then add back the triangles you construct Great ditrigonary icosidodecahedron. The octahedron has these hidden square like faces, they aren't real faces but if we make them real faces and choose only half the triangular faces you get the tetrahemihexahedron. If you look at a cuboctahedron it has a hidden hexagonal face and with the hidden hexagon faces and square faces you construct the cubohemioctahedron. But with the triangular face you construct the octahemioctahedron. Now taking the icosidodecahedron it has a hidden decagonal face and you can similarly construct the small dodecahemidodecahedron by taking the pentagons and the small icosihemidodecahedron by taking the triangles. Looking at the dodecadodecahedron it has hexagonal faces and taking just the pentagons we get the great dodecahemicosahedron. With pentagrammic faces we get the small dodecahemicosahedron. Taking the great icosidodecahedron and taking its decagonal faces and triangular faces faces we get the great icosihemidodecahedron. Taking the pentagrammic faces we get the great icosidodecahemidodecahedron.
\section*{Chapter 7: Speedround}
There is a lot more polyhedra left, but I am too lazy to subcategorize them so lets go through a ton. Truncation can be applied to the great dodecahedron and the great icosahedron. The truncated great dodecahedron has a vertex type of 10.10.5/2. The truncated icosahedron has vertex type 6.6.5/2. But what happens if you truncate a pentagram, while you get a pentagon, but wait shouldn't truncation double side count. So where are the other 5 faces in this pentagon. Lets analyze closely. As you can see the edge mysteriously merge, there are 2 edges on top of each other in this pentagon. But this face isn't dyadic, look this vertex here has 4 edges on it. So truncating the small stellated dodecahedron and the great stellated dodecahedron is not a polyhedron. But what if we replace these doubly-wound pentagons with normal pentagons. Unfortunately if we truncate the small stellated dodecahedron but replace those bad doubly-wound pentagons, we get a doubly-covered dodecahedron. We have one dodecahedron generated from the vertices and another from the faces, this again breaks dyadicity. Trying a similar process with the great stellated dodecahedron we get an icosahedron generated via the triangular vertex figures and the faces become a great dodecahedron. But as you can see here it clearly breaks dyadicity. But what if I told you that truncation is possible. See if you truncate more and more past the half way point, and even further you construct a decagram. THIS SHAPE HAS 10 SIDES. So apply this operation (known as quasi-truncation) to Kepler solids we construct the quasitruncated small stellated dodecahedron which has vertex type 10/3.10/3.5. The quasitruncated great stellated dodecahedron is a shape with vertex type 10/3.10/3.3. Now you can also quasitruncate a square forming an octagram, but you cannot quasitruncate pentagons or triangles. So if you quasi truncate a cube you construct a shape called quasitruncated hexahedron for some reason. It has a vertex type of 8/3.8/3.3. Now those are the 5 additional truncated solids. But remember the Small Rhombicuboctahedron, if you see you have octagonal pseudo faces, so if you take just the squares generated from the edges and the octagons you can construct the small rhombihexahedron, whilst if you take the squares generated from faces and the triangles from vertices you construct the small cubicuboctahedron. Now the small rhombicosidodecahedron has decagonal pseudo faces. If you choose those psuedo faces along with the pentagonal and triangular faces you construct the small dodekicosidodecahedron. Choosing the decagonal faces and the square faces you construct the small rhombidodecahedron. Now remember the small stellated dodecahedron and its dual the great dodecahedron. These shapes can be cantellated to from the rhombidodecadodecahedron. It has hexagonal pseudo faces. Use hexagonal faces along with pentagrammic and pentagonal faces we construct the icosidodecadodecahedron. But if we just choose the square faces and hexagonal faces we construct the rhombicosahedron.  Now cantellating the great stellated dodecahedron we construct a compound. This compound is of 5 cubes arranged dodecahedral symettric and one small ditrigonary icosidodecahedron.  This is why we have quasi cantellation, if we move the faces inward we construct quasirhombicosidodecahedron. Naturally this shape has facettings. Using its decagrammal psuedo faces along with the pentagrams and triangles we construct the great dodekicosidodecahedron. But ignoring the pentagrammic and triangular faces and adding in the square and decagrammic faces we construct the great rhombidodecahedron. Now quasicantellation can be applied to the cube constructing quasirhombicuboctahedron. It has octagrammic pseudo faces. Using the faces generated by squares and triangular vertex figures and octagrammic pseudo we construct the great cubicuboctahedron. But using the faces generated by edges along with octagrammic pseudo faces we construct the great rhombihexahedron. Now if you look at the truncated dodecahedron you can replace the decagonal faces with decagrams and remove the triangles. There are triangular and pentagonal pseudo faces and adding those in we construct the great ditrigonary dodekicosidodecahedron. Now removing the decagrammic faces and adding hexagonal faces we construct the great icosicosidodecahedron.  Now removing the pentagonal and triangular faces but re adding back the decagrammic faces we construct the great dodekicosahedron. Now consider a shape with a vertex type 6.5/2.3.5/2. This shape is known as a small icosicosidodecahedron. Now removing the hexagonal faces and adding decagonal faces we construct the small ditrigonary dodekicosidodecahedron. Now removing the triangular and pentagrammic faces and re adding the hexagonal faces we construct the small dodekicosahedron.  Now remember that bevel process I discussed earlier. What if instead of truncating we quasi truncate. Doing this with the cube we construct the quasitruncated cuboctahedron. Now doing this process to the small stellated dodecahedron we construct the quasitruncated dodecadodecahedron. Additionally doing this to the small stellated dodecahedron we construct the great quasitruncated icosidodecahedron. Now these 3 shapes all have 3 distinct even sided shapes around each vertex. A shape with vertex type 8.8/3.6 is called the cuboctatruncated cuboctahedron. And a shape with vertex type 10/3.10.6 is called an icosidodecatruncated icosidodecahedron. There is just one FINAL category of shapes.

\section*{Chapter 8: Snub Polyhedra}
NOTE: DO THIS SECTION.
If we take the snub of the small stellated dodecahedron and great dodecahedron we construct the snub dodecadodecahedron. Also taking the snub of the great stellated dodecahedron and great icosahedron which is a great snub icosidodecahedron. Now lets talk about the holosnub operation. Applying holosnub to a pentagon creates a pentagram, but applying a holosnub to a hexagon forms a pair of triangles. Now applying the holosnub operation to a 3d polyhedra involves taking the holosnub of all of its faces and adding in a vertex figure. If you take the holosnub of a truncated icosahedron you get a non uniform polyhedra as the vertex figure is an isoceles triangle. But truncating it a little less and then applying the holosnub we do construct a uniform shape. This is called the small snub icosicosidodecahedron. Interestingly this is the only non chiral snub polyhedra. Now interestingly if we move the faces of the small stellated dodecahedron inwards instead of outwards we can make a polyhedra called the inverted snub dodecadodecahedron. Similarly if we do this for the great stellated dodecahedron we construct the great inverted snub icosidodecahedron. The remanining polyhedra are really annoying $>:($ (5 left) 
Remember the Icosidodecatruncated icosidodecahedron alternating it and rearranging the vertices constructs the snub icosidodecadodecahedron.  Alternating the great inverted snub icosidodecahedron we get the Great inverted snub icosidodecahedron.
\newpage
\theendnotes


\end{document}

